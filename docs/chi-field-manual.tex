% Options for packages loaded elsewhere
\PassOptionsToPackage{unicode}{hyperref}
\PassOptionsToPackage{hyphens}{url}
%
\documentclass[
]{book}
\title{A Field Manual for Cultural Heritage Informatics}
\author{The X-Lab, Carleton University}
\date{2022-02-15}

\usepackage{amsmath,amssymb}
\usepackage{lmodern}
\usepackage{iftex}
\ifPDFTeX
  \usepackage[T1]{fontenc}
  \usepackage[utf8]{inputenc}
  \usepackage{textcomp} % provide euro and other symbols
\else % if luatex or xetex
  \usepackage{unicode-math}
  \defaultfontfeatures{Scale=MatchLowercase}
  \defaultfontfeatures[\rmfamily]{Ligatures=TeX,Scale=1}
\fi
% Use upquote if available, for straight quotes in verbatim environments
\IfFileExists{upquote.sty}{\usepackage{upquote}}{}
\IfFileExists{microtype.sty}{% use microtype if available
  \usepackage[]{microtype}
  \UseMicrotypeSet[protrusion]{basicmath} % disable protrusion for tt fonts
}{}
\makeatletter
\@ifundefined{KOMAClassName}{% if non-KOMA class
  \IfFileExists{parskip.sty}{%
    \usepackage{parskip}
  }{% else
    \setlength{\parindent}{0pt}
    \setlength{\parskip}{6pt plus 2pt minus 1pt}}
}{% if KOMA class
  \KOMAoptions{parskip=half}}
\makeatother
\usepackage{xcolor}
\IfFileExists{xurl.sty}{\usepackage{xurl}}{} % add URL line breaks if available
\IfFileExists{bookmark.sty}{\usepackage{bookmark}}{\usepackage{hyperref}}
\hypersetup{
  pdftitle={A Field Manual for Cultural Heritage Informatics},
  pdfauthor={The X-Lab, Carleton University},
  hidelinks,
  pdfcreator={LaTeX via pandoc}}
\urlstyle{same} % disable monospaced font for URLs
\usepackage{longtable,booktabs,array}
\usepackage{calc} % for calculating minipage widths
% Correct order of tables after \paragraph or \subparagraph
\usepackage{etoolbox}
\makeatletter
\patchcmd\longtable{\par}{\if@noskipsec\mbox{}\fi\par}{}{}
\makeatother
% Allow footnotes in longtable head/foot
\IfFileExists{footnotehyper.sty}{\usepackage{footnotehyper}}{\usepackage{footnote}}
\makesavenoteenv{longtable}
\usepackage{graphicx}
\makeatletter
\def\maxwidth{\ifdim\Gin@nat@width>\linewidth\linewidth\else\Gin@nat@width\fi}
\def\maxheight{\ifdim\Gin@nat@height>\textheight\textheight\else\Gin@nat@height\fi}
\makeatother
% Scale images if necessary, so that they will not overflow the page
% margins by default, and it is still possible to overwrite the defaults
% using explicit options in \includegraphics[width, height, ...]{}
\setkeys{Gin}{width=\maxwidth,height=\maxheight,keepaspectratio}
% Set default figure placement to htbp
\makeatletter
\def\fps@figure{htbp}
\makeatother
\setlength{\emergencystretch}{3em} % prevent overfull lines
\providecommand{\tightlist}{%
  \setlength{\itemsep}{0pt}\setlength{\parskip}{0pt}}
\setcounter{secnumdepth}{5}
\usepackage{booktabs}
\ifLuaTeX
  \usepackage{selnolig}  % disable illegal ligatures
\fi
\usepackage[]{natbib}
\bibliographystyle{apalike}

\begin{document}
\maketitle

{
\setcounter{tocdepth}{1}
\tableofcontents
}
\hypertarget{welcome}{%
\chapter{Welcome}\label{welcome}}

This field manual will help you with the immediate problems of digitization and getting your cultural heritage materials online and in a format that will help address longer term sustainability, findability, and research.

The \href{https://carleton.ca/xlab}{X-Lab} is led by \href{https://shawngraham.github.io}{Shawn Graham} with the assistance of \href{https://icmi.ca}{Monique Manatch}.

\hypertarget{about-the-xlab}{%
\section{About The XLab}\label{about-the-xlab}}

The XLab - The Cultural Heritage Informatics Collaboratory -- represents both a space and a series of relationships within and without the University. It is a kind of transdisciplinary `skunkworks' for fostering encounters with and between cultural heritage and digital media and computation. As a skunkworks, or a space for trying non-traditional or imaginative new solutions, it aims to bring together tenacious tinkerers, who are willing to experiment, to wonder, to try, and to iterate -- to try, try again. These creative engagements sometimes will not fit within existing disciplinary modes of thought or administrative structures. We accept and celebrate this -- the XLab values process and relationships, grounded in lived situations.

The digital era is, in some ways, an era that has a renewed focus on orality, on the transmission of knowledge through personal relationships. When we think of the `field' for cultural heritage informatics, we acknowledge that we are dealing with `belongings', not objects; that cultural heritage is imbued with meanings, and that knowledge holders are everyone from children to Elders, academics to artists, and that knowledge and knowledge holders are found everywhere: there are no simple binaries. Thus we see cultural heritage informatics embedded in networks of relationships, in flows of knowledge and ideas. The work of the XLab is to promote and understand these flows and relationships. The fieldwork of cultural heritage informatics can be situated in understanding the metadata of these flows.

The idea of `cultural heritage informatics' can be understood as actions in the context of relationships. Such relationships have to be fostered and built on mutual trust. These actions involve thinking through new protocols for how we work with differing communities, respecting the knowledge and sovereignty of the communities, with authentic reciprocal community engagement. It will require the creation of differentiated pathways of access that respect that not all cultural heritage knowledge is meant for everyone.

The actions of cultural heritage informatics might run the gamut from digitization and ontologies and description, to the devising of protocols and platforms for the sharing of cultural materials, to performance and storytelling cultural heritage through immersive technologies. The actions of cultural heritage informatics might be in creating the necessary metadata to link repositories of knowledge together to generate new knowledge, new relationships. It might mean the hard work of curation and restoration to decolonize collections of cultural heritage dispersed across Western museums. The actions of cultural heritage informatics, whatever they may be, are grounded in our relationships with the cultures and communities with whom we work.

\hypertarget{intro}{%
\chapter{Audience}\label{intro}}

\begin{itemize}
\tightlist
\item
  discussion here of who we are writing for; our imagined audience and their level of digital literacy
\item
  that is to say: a small scale, probably volunteer run organization with minimal access to computing resources; that is, a couple of personal computers, some smartphones, ideally a high-speed internet connection, and while webspace-of-one's-own would be good, we'll use open resources (vercel, heroku, colab, figshare, zenodo, binder, gh-pages, etc) to get things up and running
\end{itemize}

\hypertarget{how-to-use}{%
\section{How to use this book}\label{how-to-use}}

\begin{itemize}
\tightlist
\item
  how code blocks work etc
\end{itemize}

\hypertarget{set-up}{%
\section{Setting Up Your Machine}\label{set-up}}

\begin{itemize}
\tightlist
\item
  since we're imagining small scale organizations, we're also imagining the use of common low-cost PCs
\item
  this will align our goals/approaches with dh mincomp type stuff
\item
  but sometimes we'll recommend things like Binder or Reclaim Hosting, especially since Tim Sherratt has figured out how to deploy his glam workbench to reclaim cloud. The problem there is to show people how to make THAT work for their own data
\item
  installation of anaconda/miniconda; installation of R and R Studio
\item
  command line basics
\end{itemize}

\hypertarget{basic-principles}{%
\section{Basic Principles}\label{basic-principles}}

\begin{itemize}
\tightlist
\item
  future proofing
\item
  minimal computing
\item
  Indigenous Data Sovereignty \& Protocols
\item
  Open Access and Sensible Limitations
\item
  personal and community safety
\end{itemize}

\hypertarget{digitization}{%
\chapter{Digitization}\label{digitization}}

Digitization does not equal `preservation'. But it can mean that your materials find a broader audience. It can mean that you have a better sense of what materials you're responsible for, the gaps in your materials, and give you an indication of what areas you should be putting your energies into. Digitization can mean that your materials become available for research. It might even mean that your collection can be linked into other collections, so that a better, truer, picture emerges. But first things first: how does a small organization get its materials into a digital format?

\begin{itemize}
\tightlist
\item
  basic low cost digitization
\item
  data repositories available - zenodo, figshare, others - and why a small org should use them
\item
  a workflow: digitization - metadata creation - repositories - public facing - internal/external research
\end{itemize}

\hypertarget{photos}{%
\section{Photo Management}\label{photos}}

\begin{itemize}
\tightlist
\item
  digitization
\item
  Tropy for research photo management
\item
  cross reference to collectionsbuilder for putting stuff online/making it findable
\item
  IIF standards stuff
\end{itemize}

\hypertarget{scanning}{%
\section{Image-to-Text}\label{scanning}}

\begin{itemize}
\tightlist
\item
  flatbed or phone
\item
  treat as image, or treat as text?
\item
  tabula, other OCR options

  \begin{itemize}
  \tightlist
  \item
    OCR in bulk using R \& Tesseract
  \end{itemize}
\end{itemize}

\hypertarget{photogrammetry}{%
\section{3d Photogrammetry}\label{photogrammetry}}

\begin{itemize}
\tightlist
\item
  assuming conventional cameras or phones
\item
  meshroom - via app, or via commandline/google collab
\item
  hosting such models
\item
  metadata \& London Charter
\item
  photogrammetry of objects
\item
  photogrammetry of spaces
\end{itemize}

\hypertarget{digital-literacy}{%
\chapter{A Gentle Introduction to Digital Literacy}\label{digital-literacy}}

\begin{itemize}
\tightlist
\item
  digital literacy and data literacy are not the same thing
\item
  not going to `teach' python or R, but rather show working blocks of code and how to reuse/repurpose for immediate tasks
\item
  will link out to appropriate other tutorials as necessary
\item
  the basic goal is always minimal amount to get things going
\end{itemize}

\hypertarget{introduction-to-python}{%
\section{Introduction to Python}\label{introduction-to-python}}

\begin{itemize}
\tightlist
\item
  things like creating new environments for particular tasks

  \begin{itemize}
  \tightlist
  \item
    packages and where to find them
  \item
  \end{itemize}
\end{itemize}

\hypertarget{introduction-to-r-r-studio}{%
\section{Introduction to R \& R Studio}\label{introduction-to-r-r-studio}}

\begin{itemize}
\tightlist
\item
  things like creating new projects for particular tasks

  \begin{itemize}
  \tightlist
  \item
    packages and where to find them
  \item
  \end{itemize}
\end{itemize}

\hypertarget{version-control}{%
\section{Version Control}\label{version-control}}

\begin{itemize}
\tightlist
\item
  go with gitlab maybe rather than github?
\end{itemize}

\hypertarget{tutorials}{%
\chapter{Tutorials}\label{tutorials}}

\hypertarget{data-management}{%
\section{Tables to Databases}\label{data-management}}

\begin{itemize}
\tightlist
\item
  arranging data in tables
\item
  arranging data in graphs
\item
  setting up a basic database of either kind

  \begin{itemize}
  \tightlist
  \item
    sqlite, for local stuff? people have VIEWS on that
  \item
    graph basics are probably too much, for the intended audience, but maybe there's something gentle out there
  \end{itemize}
\item
  data management workflows
\end{itemize}

\hypertarget{open-refine}{%
\section{Cleaning Up Messy Data with OpenRefine}\label{open-refine}}

\hypertarget{publishing}{%
\section{Publishing Data with Datasette}\label{publishing}}

\begin{itemize}
\tightlist
\item
  putting data online with Datasette
\item
  useful plugins for Datasette
\item
  using the desktop version of Datasette
\item
  putting Datasette online but behind a password
\end{itemize}

\hypertarget{oral-interviews}{%
\section{Working With Oral Interviews}\label{oral-interviews}}

\begin{itemize}
\tightlist
\item
  automatic transcription with Mozilla DeepSpeech
\end{itemize}

\hypertarget{open-data}{%
\section{Working with Other Data Sources}\label{open-data}}

\begin{itemize}
\tightlist
\item
  accessing databases using Python, R
\item
  scraping various sites
\end{itemize}

\hypertarget{images}{%
\section{Working with Images}\label{images}}

\begin{itemize}
\tightlist
\item
  PixPlot for exploring a collection
\end{itemize}

\hypertarget{public-facing-work}{%
\chapter{Public Facing Work}\label{public-facing-work}}

\hypertarget{hugo}{%
\section{Static Websites With Hugo}\label{hugo}}

\begin{itemize}
\tightlist
\item
  your organization needs a basic, fast, secure website that is aesthetically pleasing
\end{itemize}

\hypertarget{collectionbuilder}{%
\section{Building a Collections Website with CollectionBuilder}\label{collectionbuilder}}

\begin{itemize}
\tightlist
\item
  you have wonderful materials. Showcase them.
\end{itemize}

\hypertarget{omeka}{%
\section{Omeka for Exhibitions}\label{omeka}}

\begin{itemize}
\tightlist
\item
  Tell stories about your materials, pulling them together from your collection's materials
\end{itemize}

\hypertarget{mural}{%
\section{Visual Storytelling with Mural}\label{mural}}

\begin{itemize}
\tightlist
\item
  Long-form storytelling
\end{itemize}

\hypertarget{wax}{%
\section{Exhibits with Wax}\label{wax}}

\begin{itemize}
\tightlist
\item
  why and when you might use Wax
\end{itemize}

\hypertarget{mukurtu}{%
\section{Mukurtu}\label{mukurtu}}

\begin{itemize}
\tightlist
\item
  Differing communities have differing protocols about access to cultural heritage information. Mukurtu is built with this in mind.
\end{itemize}

\hypertarget{bots}{%
\section{Bots}\label{bots}}

\begin{itemize}
\tightlist
\item
  Some simple bots that can make your materials more accessible
\end{itemize}

\hypertarget{interactive-stories}{%
\section{Interactive (Non-)Fiction}\label{interactive-stories}}

\begin{itemize}
\tightlist
\item
  Tell interactive stories using Twine or Ink or Ren'Py
\end{itemize}

\hypertarget{templates}{%
\chapter{Templates \& Other Useful Materials}\label{templates}}

On this page will be links to templates for wrangling metadata, etc.

\hypertarget{final-words}{%
\chapter{Final Words}\label{final-words}}

We have finished a nice book.

  \bibliography{book.bib,packages.bib}

\end{document}
